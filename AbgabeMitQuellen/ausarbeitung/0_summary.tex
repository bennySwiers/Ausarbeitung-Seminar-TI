% !TEX encoding = UTF-8 Unicode
\subsection*{Zusammenfassung}
\pagestyle{empty}
In der heutigen Gesellschaft kommen in vielen Bereichen Indoor-Lokalisierungssysteme zum Einsatz. Die Nischen können dabei grundverschieden sein. Neben der Feuerwehr, die Methoden zur Indoor-Lokalisierung verwendet, um Einsatzkräfte oder auch mögliche Opfer in Gebäuden ausfindig zu machen, können Lokalisierungsmethoden auch zur Überwachung von Kindern oder auch älteren Personen verwendet werden.\newline\newline
In dieser Ausarbeitung wird zu Beginn zwischen aktiven und passiven Lokalisierungsmethoden unterschieden und erläutert, was genau passive Lokalisierung ausmacht und welche Vorteile diese gegenüber aktiven Methoden haben. Im weiteren Verlauf werden verschiedene Methoden der passiven Lokalisierung vorgestellt und deren grundlegende Funktionsweise beschrieben. Es werden mögliche Anwendungsgebiete, sowie Vor- und Nachteile der jeweiligen Systeme genannt.\newline\newline Abschließend wird ein Fazit gezogen und auf mögliche Entwicklungen in diesem Forschungsbereich eingegangen. 