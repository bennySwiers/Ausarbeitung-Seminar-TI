
%% bare_conf.tex
%% V1.3
%% 2007/01/11
%% by Michael Shell
%% See:
%% http://www.michaelshell.org/
%% for current contact information.
%%
%% This is a skeleton file demonstrating the use of IEEEtran.cls
%% (requires IEEEtran.cls version 1.7 or later) with an IEEE conference paper.
%%
%% Support sites:
%% http://www.michaelshell.org/tex/ieeetran/
%% http://www.ctan.org/tex-archive/macros/latex/contrib/IEEEtran/
%% and
%% http://www.ieee.org/

%%*************************************************************************
%% Legal Notice:
%% This code is offered as-is without any warranty either expressed or
%% implied; without even the implied warranty of MERCHANTABILITY or
%% FITNESS FOR A PARTICULAR PURPOSE! 
%% User assumes all risk.
%% In no event shall IEEE or any contributor to this code be liable for
%% any damages or losses, including, but not limited to, incidental,
%% consequential, or any other damages, resulting from the use or misuse
%% of any information contained here.
%%
%% All comments are the opinions of their respective authors and are not
%% necessarily endorsed by the IEEE.
%%
%% This work is distributed under the LaTeX Project Public License (LPPL)
%% ( http://www.latex-project.org/ ) version 1.3, and may be freely used,
%% distributed and modified. A copy of the LPPL, version 1.3, is included
%% in the base LaTeX documentation of all distributions of LaTeX released
%% 2003/12/01 or later.
%% Retain all contribution notices and credits.
%% ** Modified files should be clearly indicated as such, including  **
%% ** renaming them and changing author support contact information. **
%%
%% File list of work: IEEEtran.cls, IEEEtran_HOWTO.pdf, bare_adv.tex,
%%                    bare_conf.tex, bare_jrnl.tex, bare_jrnl_compsoc.tex
%%*************************************************************************

% *** Authors should verify (and, if needed, correct) their LaTeX system  ***
% *** with the testflow diagnostic prior to trusting their LaTeX platform ***
% *** with production work. IEEE's font choices can trigger bugs that do  ***
% *** not appear when using other class files.                            ***
% The testflow support page is at:
% http://www.michaelshell.org/tex/testflow/



% Note that the a4paper option is mainly intended so that authors in
% countries using A4 can easily print to A4 and see how their papers will
% look in print - the typesetting of the document will not typically be
% affected with changes in paper size (but the bottom and side margins will).
% Use the testflow package mentioned above to verify correct handling of
% both paper sizes by the user's LaTeX system.
%
% Also note that the "draftcls" or "draftclsnofoot", not "draft", option
% should be used if it is desired that the figures are to be displayed in
% draft mode.
%
\documentclass[conference]{IEEEtran}
% Add the compsoc option for Computer Society conferences.
%
% If IEEEtran.cls has not been installed into the LaTeX system files,
% manually specify the path to it like:
% \documentclass[conference]{../sty/IEEEtran}





% Some very useful LaTeX packages include:
% (uncomment the ones you want to load)


% *** MISC UTILITY PACKAGES ***
%
%\usepackage{ifpdf}
% Heiko Oberdiek's ifpdf.sty is very useful if you need conditional
% compilation based on whether the output is pdf or dvi.
% usage:
% \ifpdf
%   % pdf code
% \else
%   % dvi code
% \fi
% The latest version of ifpdf.sty can be obtained from:
% http://www.ctan.org/tex-archive/macros/latex/contrib/oberdiek/
% Also, note that IEEEtran.cls V1.7 and later provides a builtin
% \ifCLASSINFOpdf conditional that works the same way.
% When switching from latex to pdflatex and vice-versa, the compiler may
% have to be run twice to clear warning/error messages.






% *** CITATION PACKAGES ***
%
%\usepackage{cite}
% cite.sty was written by Donald Arseneau
% V1.6 and later of IEEEtran pre-defines the format of the cite.sty package
% \cite{} output to follow that of IEEE. Loading the cite package will
% result in citation numbers being automatically sorted and properly
% "compressed/ranged". e.g., [1], [9], [2], [7], [5], [6] without using
% cite.sty will become [1], [2], [5]--[7], [9] using cite.sty. cite.sty's
% \cite will automatically add leading space, if needed. Use cite.sty's
% noadjust option (cite.sty V3.8 and later) if you want to turn this off.
% cite.sty is already installed on most LaTeX systems. Be sure and use
% version 4.0 (2003-05-27) and later if using hyperref.sty. cite.sty does
% not currently provide for hyperlinked citations.
% The latest version can be obtained at:
% http://www.ctan.org/tex-archive/macros/latex/contrib/cite/
% The documentation is contained in the cite.sty file itself.






% *** GRAPHICS RELATED PACKAGES ***
%
\ifCLASSINFOpdf
  % \usepackage[pdftex]{graphicx}
  % declare the path(s) where your graphic files are
  % \graphicspath{{../pdf/}{../jpeg/}}
  % and their extensions so you won't have to specify these with
  % every instance of \includegraphics
  % \DeclareGraphicsExtensions{.pdf,.jpeg,.png}
\else
  % or other class option (dvipsone, dvipdf, if not using dvips). graphicx
  % will default to the driver specified in the system graphics.cfg if no
  % driver is specified.
  % \usepackage[dvips]{graphicx}
  % declare the path(s) where your graphic files are
  % \graphicspath{{../eps/}}
  % and their extensions so you won't have to specify these with
  % every instance of \includegraphics
  % \DeclareGraphicsExtensions{.eps}
\fi
% graphicx was written by David Carlisle and Sebastian Rahtz. It is
% required if you want graphics, photos, etc. graphicx.sty is already
% installed on most LaTeX systems. The latest version and documentation can
% be obtained at: 
% http://www.ctan.org/tex-archive/macros/latex/required/graphics/
% Another good source of documentation is "Using Imported Graphics in
% LaTeX2e" by Keith Reckdahl which can be found as epslatex.ps or
% epslatex.pdf at: http://www.ctan.org/tex-archive/info/
%
% latex, and pdflatex in dvi mode, support graphics in encapsulated
% postscript (.eps) format. pdflatex in pdf mode supports graphics
% in .pdf, .jpeg, .png and .mps (metapost) formats. Users should ensure
% that all non-photo figures use a vector format (.eps, .pdf, .mps) and
% not a bitmapped formats (.jpeg, .png). IEEE frowns on bitmapped formats
% which can result in "jaggedy"/blurry rendering of lines and letters as
% well as large increases in file sizes.
%
% You can find documentation about the pdfTeX application at:
% http://www.tug.org/applications/pdftex





% *** MATH PACKAGES ***
%
%\usepackage[cmex10]{amsmath}
% A popular package from the American Mathematical Society that provides
% many useful and powerful commands for dealing with mathematics. If using
% it, be sure to load this package with the cmex10 option to ensure that
% only type 1 fonts will utilized at all point sizes. Without this option,
% it is possible that some math symbols, particularly those within
% footnotes, will be rendered in bitmap form which will result in a
% document that can not be IEEE Xplore compliant!
%
% Also, note that the amsmath package sets \interdisplaylinepenalty to 10000
% thus preventing page breaks from occurring within multiline equations. Use:
%\interdisplaylinepenalty=2500
% after loading amsmath to restore such page breaks as IEEEtran.cls normally
% does. amsmath.sty is already installed on most LaTeX systems. The latest
% version and documentation can be obtained at:
% http://www.ctan.org/tex-archive/macros/latex/required/amslatex/math/





% *** SPECIALIZED LIST PACKAGES ***
%
%\usepackage{algorithmic}
% algorithmic.sty was written by Peter Williams and Rogerio Brito.
% This package provides an algorithmic environment fo describing algorithms.
% You can use the algorithmic environment in-text or within a figure
% environment to provide for a floating algorithm. Do NOT use the algorithm
% floating environment provided by algorithm.sty (by the same authors) or
% algorithm2e.sty (by Christophe Fiorio) as IEEE does not use dedicated
% algorithm float types and packages that provide these will not provide
% correct IEEE style captions. The latest version and documentation of
% algorithmic.sty can be obtained at:
% http://www.ctan.org/tex-archive/macros/latex/contrib/algorithms/
% There is also a support site at:
% http://algorithms.berlios.de/index.html
% Also of interest may be the (relatively newer and more customizable)
% algorithmicx.sty package by Szasz Janos:
% http://www.ctan.org/tex-archive/macros/latex/contrib/algorithmicx/




% *** ALIGNMENT PACKAGES ***
%
%\usepackage{array}
% Frank Mittelbach's and David Carlisle's array.sty patches and improves
% the standard LaTeX2e array and tabular environments to provide better
% appearance and additional user controls. As the default LaTeX2e table
% generation code is lacking to the point of almost being broken with
% respect to the quality of the end results, all users are strongly
% advised to use an enhanced (at the very least that provided by array.sty)
% set of table tools. array.sty is already installed on most systems. The
% latest version and documentation can be obtained at:
% http://www.ctan.org/tex-archive/macros/latex/required/tools/


%\usepackage{mdwmath}
%\usepackage{mdwtab}
% Also highly recommended is Mark Wooding's extremely powerful MDW tools,
% especially mdwmath.sty and mdwtab.sty which are used to format equations
% and tables, respectively. The MDWtools set is already installed on most
% LaTeX systems. The lastest version and documentation is available at:
% http://www.ctan.org/tex-archive/macros/latex/contrib/mdwtools/


% IEEEtran contains the IEEEeqnarray family of commands that can be used to
% generate multiline equations as well as matrices, tables, etc., of high
% quality.


%\usepackage{eqparbox}
% Also of notable interest is Scott Pakin's eqparbox package for creating
% (automatically sized) equal width boxes - aka "natural width parboxes".
% Available at:
% http://www.ctan.org/tex-archive/macros/latex/contrib/eqparbox/





% *** SUBFIGURE PACKAGES ***
%\usepackage[tight,footnotesize]{subfigure}
% subfigure.sty was written by Steven Douglas Cochran. This package makes it
% easy to put subfigures in your figures. e.g., "Figure 1a and 1b". For IEEE
% work, it is a good idea to load it with the tight package option to reduce
% the amount of white space around the subfigures. subfigure.sty is already
% installed on most LaTeX systems. The latest version and documentation can
% be obtained at:
% http://www.ctan.org/tex-archive/obsolete/macros/latex/contrib/subfigure/
% subfigure.sty has been superceeded by subfig.sty.



%\usepackage[caption=false]{caption}
%\usepackage[font=footnotesize]{subfig}
% subfig.sty, also written by Steven Douglas Cochran, is the modern
% replacement for subfigure.sty. However, subfig.sty requires and
% automatically loads Axel Sommerfeldt's caption.sty which will override
% IEEEtran.cls handling of captions and this will result in nonIEEE style
% figure/table captions. To prevent this problem, be sure and preload
% caption.sty with its "caption=false" package option. This is will preserve
% IEEEtran.cls handing of captions. Version 1.3 (2005/06/28) and later 
% (recommended due to many improvements over 1.2) of subfig.sty supports
% the caption=false option directly:
%\usepackage[caption=false,font=footnotesize]{subfig}
%
% The latest version and documentation can be obtained at:
% http://www.ctan.org/tex-archive/macros/latex/contrib/subfig/
% The latest version and documentation of caption.sty can be obtained at:
% http://www.ctan.org/tex-archive/macros/latex/contrib/caption/




% *** FLOAT PACKAGES ***
%
%\usepackage{fixltx2e}
% fixltx2e, the successor to the earlier fix2col.sty, was written by
% Frank Mittelbach and David Carlisle. This package corrects a few problems
% in the LaTeX2e kernel, the most notable of which is that in current
% LaTeX2e releases, the ordering of single and double column floats is not
% guaranteed to be preserved. Thus, an unpatched LaTeX2e can allow a
% single column figure to be placed prior to an earlier double column
% figure. The latest version and documentation can be found at:
% http://www.ctan.org/tex-archive/macros/latex/base/



%\usepackage{stfloats}
% stfloats.sty was written by Sigitas Tolusis. This package gives LaTeX2e
% the ability to do double column floats at the bottom of the page as well
% as the top. (e.g., "\begin{figure*}[!b]" is not normally possible in
% LaTeX2e). It also provides a command:
%\fnbelowfloat
% to enable the placement of footnotes below bottom floats (the standard
% LaTeX2e kernel puts them above bottom floats). This is an invasive package
% which rewrites many portions of the LaTeX2e float routines. It may not work
% with other packages that modify the LaTeX2e float routines. The latest
% version and documentation can be obtained at:
% http://www.ctan.org/tex-archive/macros/latex/contrib/sttools/
% Documentation is contained in the stfloats.sty comments as well as in the
% presfull.pdf file. Do not use the stfloats baselinefloat ability as IEEE
% does not allow \baselineskip to stretch. Authors submitting work to the
% IEEE should note that IEEE rarely uses double column equations and
% that authors should try to avoid such use. Do not be tempted to use the
% cuted.sty or midfloat.sty packages (also by Sigitas Tolusis) as IEEE does
% not format its papers in such ways.





% *** PDF, URL AND HYPERLINK PACKAGES ***
%
%\usepackage{url}
% url.sty was written by Donald Arseneau. It provides better support for
% handling and breaking URLs. url.sty is already installed on most LaTeX
% systems. The latest version can be obtained at:
% http://www.ctan.org/tex-archive/macros/latex/contrib/misc/
% Read the url.sty source comments for usage information. Basically,
% \url{my_url_here}.





% *** Do not adjust lengths that control margins, column widths, etc. ***
% *** Do not use packages that alter fonts (such as pslatex).         ***
% There should be no need to do such things with IEEEtran.cls V1.6 and later.
% (Unless specifically asked to do so by the journal or conference you plan
% to submit to, of course. )


% correct bad hyphenation here
\hyphenation{op-tical net-works semi-conduc-tor}
\usepackage[utf8]{inputenc}
\usepackage[T1]{fontenc} 
\usepackage[ngerman]{babel}
\usepackage{filecontents,lipsum}
\usepackage[noadjust]{cite}
\usepackage{graphicx}
\usepackage{subfigure}
\usepackage{hyperref}
\usepackage[figure]{hypcap} 
\usepackage{float}
\usepackage{caption}

\begin{document}
%
% paper title
% can use linebreaks \\ within to get better formatting as desired
\title{Absicherung des Internets der Dinge}


% author names and affiliations
% use a multiple column layout for up to three different
% affiliations
\author{\IEEEauthorblockN{Artemij Olegovic Voskobojnikov}
\IEEEauthorblockA{Freie Universität Berlin\\Fakultät für
Mathematik und Informatik\\
Berlin, Germany\\
Email: voskobojnikov.artemij@gmail.com}
\and
\IEEEauthorblockN{Benjamin Swiers}
\IEEEauthorblockA{Freie Universität Berlin\\Fakultät für
Mathematik und Informatik\\
Berlin, Germany\\
Email: swiers.benjamin@googlemail.com}}

% conference papers do not typically use \thanks and this command
% is locked out in conference mode. If really needed, such as for
% the acknowledgment of grants, issue a \IEEEoverridecommandlockouts
% after \documentclass

% for over three affiliations, or if they all won't fit within the width
% of the page, use this alternative format:
% 
%\author{\IEEEauthorblockN{Michael Shell\IEEEauthorrefmark{1},
%Homer Simpson\IEEEauthorrefmark{2},
%James Kirk\IEEEauthorrefmark{3}, 
%Montgomery Scott\IEEEauthorrefmark{3} and
%Eldon Tyrell\IEEEauthorrefmark{4}}
%\IEEEauthorblockA{\IEEEauthorrefmark{1}School of Electrical and Computer Engineering\\
%Georgia Institute of Technology,
%Atlanta, Georgia 30332--0250\\ Email: see http://www.michaelshell.org/contact.html}
%\IEEEauthorblockA{\IEEEauthorrefmark{2}Twentieth Century Fox, Springfield, USA\\
%Email: homer@thesimpsons.com}
%\IEEEauthorblockA{\IEEEauthorrefmark{3}Starfleet Academy, San Francisco, California 96678-2391\\
%Telephone: (800) 555--1212, Fax: (888) 555--1212}
%\IEEEauthorblockA{\IEEEauthorrefmark{4}Tyrell Inc., 123 Replicant Street, Los Angeles, California 90210--4321}}




% use for special paper notices
%\IEEEspecialpapernotice{(Invited Paper)}




% make the title area
\maketitle


\begin{abstract}
Aktuellen Prognosen zufolge werden im Jahre 2020 rund 50 Milliarden "Dinge" mit dem Internet verbunden sein. Die Verknüpfung der virtuellen mit der dinglichen Welt stellt die Entwickler derartiger Systeme zukünftig vor diverse Herausforderungen, welche insbesondere die sichere technische Umsetzung von Datenübertragung und Datenspeicherung beinhaltet. In dieser Ausarbeitung wird das Konzept des \textit{Internet of Things} (\textit{IoT}, z.Dt. \textit{Internet der Dinge}) erläutert und auf potentielle Risiken eingegangen, welche eine derart vernetze Welt mit sich bringt. Weiterhin werden Lösungsansätze für Sicherheitsmechanismen, sowohl im privaten als auch im industriellen Bereich, genannt. \hfill Juli, 2015 
\end{abstract}
% IEEEtran.cls defaults to using nonbold math in the Abstract.
% This preserves the distinction between vectors and scalars. However,
% if the conference you are submitting to favors bold math in the abstract,
% then you can use LaTeX's standard command \boldmath at the very start
% of the abstract to achieve this. Many IEEE journals/conferences frown on
% math in the abstract anyway.

% no keywords




% For peer review papers, you can put extra information on the cover
% page as needed:
% \ifCLASSOPTIONpeerreview
% \begin{center} \bfseries EDICS Category: 3-BBND \end{center}
% \fi
%
% For peerreview papers, this IEEEtran command inserts a page break and
% creates the second title. It will be ignored for other modes.
\IEEEpeerreviewmaketitle



\section{Einleitung}
% no \IEEEPARstart
Die Gegenwart von Computern hat Einzug in unseren Alltag gehalten. Während noch vor einigen Jahren ein stationärer Computer für das gelegentliche Surfen im Internet ausreichte, begleiten uns heute Smartphones, Tablets uns Smart Watches auf all unseren Wegen. Das Mitführen solcher Geräte erfüllt dabei nicht nur den Zweck, jederzeit erreichbar zu sein, sondern ermöglicht dem Nutzer weiterhin jederzeit auf das Internet zuzugreifen. Durch diese permanente Anbindung an das Internet entstehen eine Vielzahl von Gefahren, welchen jeder Nutzer potentiell ausgesetzt ist. \\
Unsere täglichen Begleiter beinhalten unzählige vertrauliche Daten, seien es Medien, Kontaktdaten oder die Inhalte eines Terminkalenders. All diese Informationen befinden sich oft nicht nur auf den Geräten selbst, sondern mitunter auf Cloud-Systemen, wie iCloud, Dropbox oder Google+. Dafür gibt es mehrere Gründe. Zum einen verfügen diese Geräte hinsichtlich Speicherkapazität, Rechenleistung und Energie über begrenzte Ressourcen. Darüber hinaus ermöglicht die Auslagerung der Daten auf die genannten Systeme, diese Daten mit anderen Geräten zu synchronisieren. \\
Betrachtet man hinsichtlich der zur Verfügung stehenden Ressourcen die Leistungsfähigkeit von kleinen Chips, wie sie in Bereichen der Hausautomation und der Industrie verwendet werden, so wird klar, dass diese oft über weitaus weniger Speicherkapazität, Rechenleistung und Energie verfügen. \\
Um ein möglichst hohes Maß an Sicherheit für \textit{Iot}-Geräten zu gewährleisten, muss also ein Weg gefunden werden, der die permanente Verfügbarkeit von Information und die sichere Übertragung dieser Information auf Geräten mit begrenzten Ressourcen ermöglicht. Im Verlauf dieser Ausarbeitung werden Lösungsansätze aufgezeigt, welche zur Umsetzung dieser Anforderung im \textit{Internet der Dinge} beitragen.

% You must have at least 2 lines in the paragraph with the drop letter
% (should never be an issue)

 


% !TEX encoding = UTF-8 Unicode

\section{Das Internet der Dinge}
Bevor auf Lösungen zur Optimierung der Sicherheit im \textit{IoT} eingegangen wird, soll zunächst geklärt werden, was das \textit{Internet der Dinge} genau ist und welche Eigenschaften sich hinter diesem Begriff verbergen. In \cite{fried} wird das \textit{Internet der Dinge} wie folgt beschrieben: Das \textit{Internet der Dinge} verfolgt "[...] das Ziel einer Unterstützung des Menschen sowie einer durchgängigen Optimierung wirtschaftlicher und sozialer Prozesse durch eine Vielzahl von in die Umgebung eingebrachten Mikroprozessoren und Sensoren." Weiterhin ergeben sich daraus laut \cite{fried} folgende Eigenschaften:
\begin{itemize}
  \item \textit{Dezentralität bzw. Modularität}: \\
  \textit{IoT}-Geräte sind modular aufgebaut und lassen sich miteinander kombinieren, sodass sie miteinander kommunizieren und interagieren. \newline
  \item \textit{Einbettung}: \\
  Da die verwendeten Geräte zunehmend kleiner und portabler werden, können diese in Gegenstände des Alltags integriert werden, wodurch ein \textit{IoT}-Gerät mitunter nicht mehr als solches Erkennbar ist (z.B. ein RFID-Chip in der Kleidung). \newline
  \item \textit{Mobilität}: \\
  \textit{IoT}-Geräte müssen in der Lage sein, sich den Umgebungen des Nutzers anzupassen, um zu jeder Zeit an jedem Ort den Funktionsumfang zu gewährleisten, für den das entsprechende Gerät konzipiert wurde. \newline
  \item \textit{(Spontane) Vernetzung}: \\
  Ein wesentliches Merkmal des Internets der Dinge ist die Tatsache, dass die Geräte über das Internet oder anderen Netzwerktechnologien miteinander verbunden sind. Dabei ist es denkbar, dass die Geräte, je nach Einsatzgebiet, spontan (ad hoc) eine Verbinden zueinander aufnehmen können. Ein Beispielszenario wäre die in \cite{adhoc} beschrieben Kommunikation von Autos, die sich auf einer Straße entgegenkommen, um den Verkehrsstatus der kommenden Wegstrecke untereinander auszutauschen oder auf mögliche Gefahren hinzuweisen. \newline
  \item \textit{Kontextsensitivität}: \\
  \textit{IoT}-Systeme sammeln Informationen ihrer Umgebung und passen die angebotenen Dienste an die Bedürfnisse des Nutzersan den jeweiligen Kontext an.  \newline
  \item \textit{Autonomie}: \\ 
  Das System reagiert eigenständig (ohne den Zugriff des Nutzers) auf bestimmte Ereignisse. \newline
  \item \textit{Energieautarkie}:\\ 
 Da sich das Einsatzgebiet eines \textit{IoT}-Gerätes auf viele verschiedene Bereiche erstreckt, muss dieses über eine eigene Energiequelle verfügen und sollte nicht auf die stationäre Versorgung angewiesen sein.  
\end{itemize}
Die Einsatzgebiete von \textit{IoT}-Geräten Geräte sind vielfältig. Sie können in eine Infrastruktur (Gegenstände, Gebäude) und in andere mobile Geräte (Mobiltelefone, Wearables, Accessoires) eingebettet oder sogar dem Nutzer selbst implantiert werden (computerisierte Implantate)\cite{fried}. \\

\section{Sicherheit im IoT}
\subsection{Sicherheit von existierenden IoT-Geräten}
Mit der wachsenden Zahl der \textit{IoT}-Geräte steigt ebenfalls die Angriffsfläche für potentielle Angreifer. Da das \textit{Internet der Dinge} noch vergleichsweise jung ist, sind viele Geräte, die bereits auf dem Markt erhältlich sind noch nicht bis ins Detail ausgereift. Viele Techniken und Methoden, die sich in der Informationstechnik bereits seit langem bewährt haben, können auf Grund der beschränkten Ressourcen in \textit{IoT}-Geräten nicht verwendet werden. Einen erschreckenden Einblick in den Sicherheitsstand von im Handel erhältliche Geräten gewährt \cite{hpack}. \\
In dem Report wurden 10 führende \textit{IoT}-Geräte, darunter Fernseher, Webcams, Thermostate, fernsteuerbare Steckdosen, Löschanlagen, Türschlösser, Waagen und Garagenöffner auf sicherheitsrelevante Schwachstellen untersucht. Bei den 10 Geräten wurden insgesamt 250 Schwachstellen gefunden. Die \autoref{fig:hpfind} zeigt die prozentuale Verteilung verschiedener Sicherheitslücken in den Geräten.   

\begin{figure}[h]
\centering
  \includegraphics[width=1.0\columnwidth]{hp}
  \caption{Reseach findings}
  \label{fig:hpfind}
\end{figure}

\subsection{Hauptgründe für Sicherheitsprobleme im IoT}
Betrachtet man die Ergebnisse aus \cite{hpack}, so lassen sich einige Probleme kategorisieren und zu bestimmten Problemklassen zusammenführen. In \cite{owasp} werden folgende 10 Hauptgründe für Sicherheitsprobleme im \textit{IoT} genannt:
\begin{itemize}
  \item[a] \textit{Unsichere Web Interfaces} 
  \item[b] \textit{Unzureichende Authentifikation} 
  \item[c] \textit{Unsichere Netzwerk Services} 
  \item[d] \textit{Mangelnde Transportverschlüsselung} 
  \item[e] \textit{Datenschutz/ Privatsphäre} 
  \item[f] \textit{Unsichere Cloud-Schnittstellen} 
  \item[g] \textit{Unsichere Mobile-Interfaces} 
  \item[h] \textit{Unzureichendes Maß an Sicherheitskonfiguration} 
  \item[i] \textit{Unsichere Software/ Firmware} 
  \item[j] \textit{Mangelnde physische Sicherheit} \\
\end{itemize} 
Im Folgenden werden diese Probleme erläutert, mögliche Angriffe aufgezeigt und Lösungen zur Reduzierung der Sicherheitsprobleme genannt. \\

\subsubsection{Unsichere Benutzer-Schnittstellen} 
Die Probleme \textit{a}, \textit{f} und \textit{g} betreffen die User-Interfaces, welche die Schnittstelle zwischen einem Nutzer und der Anwendung darstellen. Wenn Benutzer-Schnittstellen nicht ausreichend sicher implementiert werden, so sind Angriffe, wie \textit{Account Enumeration} (Herausfinden, ob ein bestimmter Benutzername existiert), \textit{Cross-site Scripting, XSS} (Ausführen von bösartigem Code in vertrauenswürdigem Context) oder \textit{SQL-Injections} (Eingabe bösartiger Nutzereingaben, welche die Manipulation der Daten in der Datenbank zur Folge haben) möglich. \\
Weiterhin birgt die Verwendung von schwachen Anmeldedaten die Gefahr, dass ein Angreifer, mit einem zu geringen Aufwand Zugriff auf ein Benutzerkonto bekommt, für das dieser eigentlich nicht autorisiert ist.  \\ 
Um diesen Angriffen zuvorzukommen sollten Entwickler die Eingaben der Schnittstellen validieren und einen robusten Passwort-Recovery-Mechanismus verwendet, durch den ein potentieller Angreifer keinen Hinweis auf einen gültigen Account bekommt\cite{owasp}. Wenn ein Benutzer (eventuell Angreifer) zu oft falsche Anmeldedaten eingibt, so sollte dies eine Sperrung des Benutzerkontos zur Folge haben. Darüber hinaus dürfen die Anmeldedaten nicht offen dargelegt werden (Netzwerkverkehr). \\
\subsubsection{Unzureichende Authentifikation}
Ein wesentlicher Bestandteil für den sicheren Zugang zu einem System stellt der verwendete Authentifikationsmechanismus dar. Eine unzureichende Passwortkomplexität ermöglicht Angreifern den Zugang zum System, wodurch er dieses direkt manipulieren oder Zugang zu schutzbedürftigen Daten erlangen kann. \\
Durch die Verwendung von entsprechend komplexen Anmeldedaten und die Verschlüsselung dieser Daten kann ein weitaus höheres Maß an Sicherheit gewährleistet werden. Wenn möglich, sollte eine \textit{Zwei-Faktor-Zwei-Faktor-Authentifizierung} oder eine \textit{Public Key Infrastructur (PKI)} genutzt werden\cite{owasp}. \\
\subsubsection{Unsichere Netzwerk Services}
Da die Bestandteile einer \textit{IoT}-Infrastruktur alle miteinander vernetzt sind, bedeutet dies, dass ständig Daten miteinander ausgetauscht werden. Der Zugang zum System beinhaltet ein hohes Gefahrenpotential, denn gelingt es einem Angreifer sich über die Netzwerkservices einen Zugang zum System zu verschaffen, so kann dieser  die übermittelten Daten abfangen, das System manipulieren oder funktionsunfähig zu machen (z.B. durch einen \textit{Denial-of-Service}-Angriff). Die Schwachstelle bilden hier unnötig offene Ports oder angreifbare Services. \\
Um diesen Gefahren vorzubeugen, sollten die zu übermittelten Daten stets verschlüsselt übertragen werden. Weiterhin sollten nur diejenigen Ports erreichbar sein, die für die Aufrechterhaltung der Funktionsweise des Systems notwendig sind. Bestimmte Ports und Services sollten nicht über das Internet, z.B. mittels \textit{Universal Plug and Play (UPnP)} erreichbar sein\cite{owasp}. Die Herstellern und Entwicklern müssen in diesem Punkt einen Kompromiss finden zwischen Erreichbarkeit und Angreifbarkeit. \\
\subsubsection{Mangelnde Transportverschlüsselung}
Auch die Verwendung einer mangelnden Transportverschlüsselung führt potentiell zur Verletzung aller Schutzziele (Vertraulichkeit, Integrität, Verfügbarkeit). Ein wesentlicher Aspekt bildet hier die Nutzung geeigneter Protokolle(\textit{Lightweight Kryptography})\cite{owasp}, welche die verschlüsselte Übertragung der Daten mit begrenzten Ressourcen ermöglichen. Beispiele für mögliche Protokolle und Verfahren sind das \textit{Constrained Application Protocol (CoAP)} und \textit{ZigBee}, aber auch Standards wie \textit{Bluetooth} oder \textit{Near Field Communication (NFC)}, welche durch ihre geringe Reichweite ein gewisses Maß an Sicherheit gewährleisten\cite{lowpow}\cite{zigbee}.\\
\subsubsection{Datenschutz/ Privatsphäre}
Im Internet der Dinge werden unzählige Daten zwischen den Geräten ausgetauscht. Viele dieser Daten beinhalten personenbezogene und vertrauliche Information. Würden diese in die Hände eines Angreifers gelangen, so würde dies zur Verletzung der informationellen Selbstbestimmung führen. Aus diesem Grund sollte die Berücksichtigung von Datenschutz-Aspekten bereits bei der Entwicklung eines System bedacht werden und nicht erst im Nachhinein eingebaut werden(\textit{Privacy by Design})\cite{seciot}. Diese Systeme sollten unter dem Aspekt der Datensparsamkeit bzw. Datenvermeidung agieren und, wenn möglich, die Anonymisierung oder Pseudonymisierung de Daten in Betracht ziehen. Weiterhin sollte nicht mehr benötigte Daten gelöscht werden. 
\subsubsection{Unzureichendes Maß an Sicherheitskonfiguration}
Ein System wird oft von verschiedenen Nutzern verwendet und bedient. Aus diesem Grund sollte es ein feingranulare Trennung verschiedener Rollen geben, mit denen ein Benutzer dem System gegenübertritt. Nur so kann gewährleistet werden, dass ein einfacher Nutzer nicht Funktionalitäten nutzen kann, die eigentlich einem Systembetreuer oder Ähnlichem obliegen. \\
Weiterhin sollte ein Monitoring und Logging von sicherheitsrelevanten Ereignissen durchgeführt werden, um mögliche Schwachstellen und Angriffe zu bemerken\cite{owasp}. 
\subsubsection{Unsichere Software/ Firmware}
Da jedes \textit{Iot}-Gerät durch entsprechende Software funktioniert, muss sichergestellt sein, dass diese möglichst aktuell gehalten wird. Dabei ist zu beachten, dass das einspielen von Updates wiederum ein Sicherheitsrisiko birgt. Sollten die Updates unverschlüsselt übertragen werden, so kann ein Angreifer die Daten verändern und so zu seinen Gunsten nutzen\cite{owasp}. \\
Aus diesem Grund sollten auch Updates stets verschlüsselt übertragen werden. Darüber hinaus sollte dieses signiert und zertifiziert sein.
\subsubsection{Mangelnde physische Sicherheit}
Ein wichtiger und oft vernachlässigter Aspekt ist die physische Sicherheit der Komponenten. Wenn diese einfach zugänglich sind, kann eine Manipulation der Geräte nie ausgeschlossen werden. Ein Speichermedium sollte nicht leicht entfernbar sein und das Auseinanderbauen eines Gerätes sollten sollte sich möglichst schwierig gestalten\cite{owasp}.\\

\section{Smart Home}
Der Einsatz von \textit{IoT}-Geräten zur Vernetzung verschiedener Komponenten in Wohngebäuden wird als \textit{Smart Home} bezeichnet\cite{andelf}. 

\begin{figure}[h]
\centering
  \includegraphics[width=0.7\columnwidth]{smarthome}
  \caption{Anwendungsbereiche im Smart Home}
  \label{fig:smarthome}
\end{figure}

Die \autoref{fig:smarthome} zeigt die Anwendungsbereiche für einen vernetzen Haushalt.
Um ein möglichst sicheres Umfeld für den Heimgebrauch zu schaffen, sollten ausschließlich geeignete Hardware verwendet werden. Zum einen sind dies Bauteile, die sogenannte \textit{PUFS} (siehe Abschnitt Sichere Komponenten) verwenden und darüber hinaus Bauteile, welche auf äußere Einflüsse reagieren. Diese Bauteile erkennen Gefahren (unberechtigter Fremdzugang, Manipulation, Sabotage) und veranlassen beispielsweise die Trennung vom Internet\cite{landkrim}. \\
Weiterhin sollte die leitungsgebundene Übertragung dem WLAN oder anderen Funk-Übertragungswegen vorzuziehen sein. Dieses Vorgehen schließt den Zugriff eines Angreifers nicht gänzlich aus, jedoch muss ein solcher einen höheren Aufwand betreiben, um Zugang zum System zu erhalten. \\
Auch im \textit{Smart Home}-Bereich sollte besonderes Augenmerk auf die Verwendung von entsprechend komplexen Passwörtern für den Zugang zum System und die Nutzung von Verschlüsselungstechniken nach dem Stand der Technik (z.B. \textit{AES}) gelegt werden. Weiterhin empfiehlt sich die Benutzung von Softwarekomponenten, welche laufende Angriffe erkennen (\textit{Intrusion Detektion}). Dies kann durch Firewalls mit Application Filtern realisiert werden.   

\section{Anwendung des IoT in der Industrie}

In Unternehmen wird meist zwischen zwei Informations- und Kommunikationstechnikbereichen unterschieden. Zum einen existiert in Unternehmen die \textit{Produktions-IT}, welche jegliche Controller von Produktionsanlagen oder Logistikanbindungen umfasst. Zum anderen gibt es ebenfalls den Bereich der \textit{Business-IT}. Darunter fallen jegliche Anwendungen, die von den Angestellten verwendet werden und in keinem direkten Zusammenhang mit der Produktion stehen. Ein Beispiel dafür wären \textit{Content-Management-Systeme}.\\

\section{Industrie 4.0}
Die \textit{Industrie 4.0} vernetzt beide Bereiche miteinander und als Resultat entstehen sogenannte \textit{cyberphysikalische Systeme} (\textit{CPS}).\\
Bei einem \textit{CPS} handelt es sich meist um ein Gerät mit beschränkten Ressourcen. Dies bedeutet, dass diese Systeme in der Rechenleistung und im Energieverbrauch beschränkt sind. Zusätzlich gibt es Anforderungen an solche Systeme, die unbedingt erfüllt werden müssen. So müssen \textit{CPS} ständig verfügbar und ausfallsicher sein, sodass es im schlimmsten Fall nicht zum Produktionsstillstand kommen kann.
\subsection{Anforderungen an derartige Systeme}
Aufgrund der Vernetzung beider \textit{IKT-Bereiche}, werden Anforderungen des jeweiligen Subsystems übernommen. Dies bedeutet, dass Sicherheitslösungen im Office-Bereich nicht ohne weiteres angewendet werden können, da in einem \textit{CPS} auch Auswirkungen auf den anderen IKT-Bereich (in diesem Beispiel die \textit{Produktions-IT}) abgeschätzt werden müssen. Aus dieser Problematik resultieren somit neue Probleme und Schwachstellen, die möglicherweise bei den anfänglich separierten Systemen nicht existierten.\\
Im Detail bedeutet das, dass sich insbesondere die Sicherheitsanforderungen unterscheiden und bekannte Sicherheitslösungen der \textit{Business-IT} wie VPN oder SSL/TLS-Verschlüsselung nicht auf die \textit{Produktions-IT} übertragen werden können.\\
Der Hauptgrund dafür ist die Tatsache, dass Komponenten der \textit{Produktions-IT} zertifiziert sind und eine Verwendung von Verschlüsselungen einen Eingriff bedeuten würde, welcher im schlimmsten Fall zu Verlust der Zertifizierungen führen würde. Ebenfalls können Verschlüsselungen zu möglichen Latenzen führen, die im Office-Bereich verkraftbar sind, in einer Produktion aber eine Nichtfunktionalität bedeuten würden.\\
Ein weiterer Aspekt, der in der \textit{Business-IT} nicht bedacht werden musste, sind physikalische Angriffe, die in der Produktion denkbar sind. So können Mitarbeiter Komponenten verändern oder gänzlich entfernen \cite{eckert2015}.     
\vspace{0.3cm}
\\
In den folgenden Abschnitten werden mögliche Probleme der \textit{Industrie 4.0} aufgezeigt und es werden Lösungsansätze demonstriert, die aktuell im Einsatz sind.

\subsection{Cloud-Computing}

\subsubsection{Unterarten von Clouds}
Cloud-Computing ist nicht gleich Cloud-Computing. Es gibt verschiedene Services, die auf verschiedene Anwendergruppen zugeschnitten sind, wie die nachfolgende Grafik demonstriert.\\

\begin{figure}[h]
\centering
  \includegraphics[width=0.7\columnwidth]{clouds}
  \caption{Unterarten von Clouds}
  \label{fig:soc}
\end{figure}

Die umgekehrte Pyramide verdeutlicht dabei den unterschiedlichen Umfang des jeweiligen Dienstes.
Dabei wird zwischen drei verschiedenen Services unterschieden. Bei \textit{Software-as-a-Service} (SaaS) werden vollständige Systeme den Nutzern zur Verfügung gestellt. Der Zugriff auf diese erfolgt mittels Browser.\\
\textit{Product-as-a-Service} (PaaS) ist in der Regel auf Anwendungsentwickler bzw. fortgeschrittene Nutzer zugeschnitten, welche möglicherweise eigene Applikationen auf den Systemen verwenden wollen. Die letzte Unterart \textit{Infrastructure-as-a-Service} (IaaS) bietet dem Anwender lediglich ein Grundgerüst \cite{channel2015}. Es können personalisierte Dienste oder Betriebssysteme installiert werden. Große Unternehmen wählen häufig IaaS als Cloud-Lösung, weil diese eine komplette Plattform bieten und ein Drittanbieter für die Wartung dieser zuständig ist. Darüber hinaus ist der Drittanbieter im Falle eines Angriffs auch verantwortlich.

\subsection{Industrie und die Cloud}
\begin{figure}[h]
\centering
  \includegraphics[width=0.4\textwidth]{firmen}
  \caption{Clouds in der Industrie}
  \label{fig:clouds}
\end{figure}

Aufgrund der Verwendung von beschränkten Ressourcen wird oftmals auf \textit{cloudbasierte Lösungen} industriellen im industriellen Bereich zurückgegriffen, wie die \autoref{fig:clouds} aufzeigt. Diese Verlagerung bedeutet, dass Rechenoperationen sowie die gesamte Datenspeicherung nicht auf dem limitierten Gerät selbst durchgeführt werden muss, es stattdessen vielmehr mit der Cloud in ständiger Kommunikation steht.\\
Trotz der verbreiteten Verwendung, sind Cloudlösungen nicht durchgehend akzeptiert. Dies ist der Statistik in \autoref{fig:bedenken} zu entnehmen. Dabei handelt es sich um eine Statistik, die bei der \textit{Pressekonferenz Cloud Monitor 2015} präsentiert wurde \cite{bitkom}. Die größten Bedenken beziehen sich auf die Sicherheit der Cloud, insbesondere auf Schutz der sensiblen (personenbezogenen) Daten. Aus diesem Grund wird im Folgenden auf die Absicherung dieser eingegangen.
\begin{figure}[h]
\centering
  \includegraphics[width=0.4\textwidth]{bedenken}
  \caption{Sicherheitsbedenken der Nutzer}
  \label{fig:bedenken}
\end{figure}

\subsection{Trusted Cloud}

Bei der \textit{Trusted Cloud} handelt es sich um ein Programm des Bundesministeriums für Wirtschaft und Energie. Das Ziel dieses Projektes war es die Entwicklung einer sicheren Cloud-Infrastruktur, die böswilligen Administratoren den Zugriff auf sensible Daten verwehren soll. \cite{windriver}\\
Das Grundkonzept sind speziell gesicherte Serverschränke, die bei unberechtigten Eingriffen sich herunterfahren. Zusätzlich werden jegliche Daten verschlüsselt auf den Festplatten gespeichert. Bei der Verarbeitung der Daten wird darüber hinaus zusätzlich nur flüchtiger Speicher verwendet, sodass nach Herunterfahren des Systems auf keinerlei Daten zugegriffen werden kann \cite{eckert2015}.

\subsection{Zugriffskontrolle bei Cloud-Diensten}

Aufgrund der Vielzahl von verschiedenen Nutzern bei einem Cloud-Dienst muss ebenfalls der Zugriff auf die Datensätze abgesichert werden. Dies könnte mittels einer \textit{rollenbasierten Zugriffskontrolle} geschehen, doch besteht hierbei der Nachteil, dass ein möglicher Fehler in der Implementierung zur Aufhebung der Rollen führen würde. Dies ist nur möglich, weil solche Systeme im Regelfall vollständig in Software umgesetzt werden.\\
Eine bessere Möglichkeit bietet ein \textit{attributbasiertes Verschlüsselungsystem} (s. \autoref{fig:attrversch}).  
\begin{figure*}
\centering
\includegraphics[width=1.5\columnwidth]{attr}
\caption{Attributbasiertes Verschlüsselungssystem}
\label{fig:attrversch}
\end{figure*}

Zugrunde liegt hier ein \textit{asymmetrisches Verschlüsselungsverfahren}. Die \textit{Public Keys} sowie \textit{Private Keys} werden dabei vom \textit{Private Key Generator} ausgestellt.\\
In der Praxis würde also nur eine Person zugreifen können, die das nötige Attribut besitzt \cite{eckert2015}.\\
Bei mehreren Anwendern könnte man eine gruppenbasierte Zugriffskontrolle einführen. Dafür kann beispielsweise der Verzeichnisdienst \textit{Active Directory} (AD) verwendet werden. Mit Hilfe von AD können nun Gruppen erstellt werden und digitale Zertifikate gruppenweit verteilt werden. Eine Zugriffskontrolle für mehrere Nutzer wäre damit gewährleistet.\\

\subsection{Suchen auf verschlüsselten Daten}
Aufgrund der Tatsache, dass die Daten verschlüsselt auf den Datenträgern vorliegen, müssen spezielle Algorithmen verwendet werden, die eine Suche auf verschlüsselten Datensätzen ermöglichen. Dies ist notwendig, da so die Daten zu jedem Zeitpunkt verschlüsselt auf dem Server vorliegen und nur clientseitig entschlüsselt werden. Böswillige Administratoren haben so im besten Fall keine Angriffsmöglichkeit (serverseitig).\\

\subsection{Secure Indexes}
In dieser Ausarbeitung beschränken wir uns auf \textit{Symmetric Searchable Encryption} (SSE), also auf Verschlüsselungsverfahren, die auf einem symmetrischen Schlüssel/Schema basieren. Ein Index ist dabei eine Datenstruktur, die bei Eingabe eines Teilwortes die Pointer zu den Dokumenten wiedergibt, in denen das Teilwort enthalten ist \cite{sse}. Zusätzlich kann man einen Index als sicher ansehen, wenn die Suche nach einem Teilwort $w$ nur mit Hilfe einer \textit{Trapdoor} durchgeführt werden kann und ein Index alleine keinerlei Auskunft über den Inhalt gibt. Die Generierung einer sogenannten \textit{Trapdoor} erfolgt dabei mittels des privaten Schlüssels des Nutzers. Im Folgenden wird eine beispielhafte \textit{SSE} nach Goh \cite{gohindex} erläutert:\\
Zu Beginn muss der Begriff der \textit{Bloom-Filter} eingeführt werden, da diese existenziell für die Erstellung der sicheren Indizes sind. Bei einem \textit{Bloom-Filter} handelt es sich um eine Datenstruktur, welche eine Menge $S = \{s_1,...,s_n\}$ mit $n$ Elementen repräsentiert. Solch ein Filter besteht aus einem $m$-stelligen Bitarray, in welchem jede Stelle anfangs auf $0$ gesetzt wird. Anschließend werden $r$ unabhängige Hashfunktionen $h_1,...,h_r$ gewählt, wobei $h_i : \{0,1\}^* \rightarrow [1,m]$ für $i \in [1,r]$ gilt. Zusätzlich werden für jedes Element $s \in S$ im $m$-stelligen Array die Positionen $h_1(s),...,h_r(s)$ auf $1$ gesetzt.\\
Die für die weitere Betrachtung relevante Operation ist nun das Prüfen, ob ein Element $a$ in der Menge $S$ ist. Hierfür werden die Positionen $h_1(a),...,h_r(a)$ geprüft und falls all diese Positionen eine $1$ sind, gehört $a$ zur Menge $S$ \cite{bloom}.\\
In dem folgenden Abschnitt wird die Konstruktion von \textit{sicheren Indizes} nach \cite{gohindex} erklärt.

\subsection{Konstruktion nach Goh \cite{gohindex}}

\begin{itemize}
  \item \texttt{Keygen(s)}: Ein Sicherheitsparameter $s$ wird gegeben. Eine pseudozufällige Funktion $f: \{0,1\}^n \times \{0,1\}^s \rightarrow \{0,1\}^s$ wird gewählt sowie ein privater Schlüssel $K_{priv} = (k_1,...,k_r) \leftarrow \{0,1\}^{sr}$ 
  \item \texttt{Trapdoor($K_{priv}$, $w$)}: Eingabe für die \textit{Trapdoor} sind der private Schlüssel $K_{priv}$ und das Wort $w$, die Ausgabe für die beiden Parameter ist $T_w = (f(w,k_1),...,f(w,k_r)) \in \{0,1\}^{sr}$
  \item \texttt{BuildIndex($D$, $K_{priv}$)}: Eingabe für die Funktion \texttt{BuildIndex} ist das Dokument $D$ mit einer einzigartigen ID $D_{id} \in \{0,1\}^n$ und einer Liste von Worten $(w_0,...,w_t) \in \{0,1\}^{nt}$ sowie dem privaten Schlüssel $K_{priv} = (k_1,...,k_n) \in \{0,1\}^{sr}$
  Für jedes einzigartige Wort $w_i$, wobei $i \in [0,t]$ werden folgende drei Berechnungen angestellt
  \begin{enumerate}
    \item Berechnung der \textit{Trapdoor}: $(x_1=f(w_i,k_i),...,x_r=f(w_i,k_r)) \in \{0,1\}^{sr}$
    \item Berechnung eines eindeutigen Codewortes für jedes $w_i$: $(y_1 = f(D_{id}, x1),...,y_r = f(D_{id},xr)) \in \{0,1\}^{sr}$
    \item Abschließend wird das Codewort $y_1,...,y_r$ in den \textit{Bloom-Filter} des Dokumentes mit der $D_{id}$ eingefügt
  \end{enumerate}
  Der Output ist der Index $I_{D_{id}} = (D_{id}, BF)$ für das Dokument mit der ID $D_{id}$
  \item \texttt{SearchIndex($T_w$, $I_D$)}: Der Input ist die \textit{Trapdoor} $T_w = (x_1,...,x_r) \in \{0,1\}^{sr}$ für das jeweilige Wort $w$ und ein Index $I_D$ mit $I_{D{id}} = (D_{id}, BF)$ für das jeweilige Dokument mit der ID $D_{id}$. Es müssen die folgenden Berechnungen durchgeführt werden:\\
  \begin{enumerate}
    \item Berechne das Codewort für das Wort $w$ in $D_{id}: (y_1 = f(D_{id}, x_1),...,y_r = f(D_{id},xr)) \in \{0,1\}^{s,r}$
    \item Wenn der \textit{Bloom-Filter} nur Einsen an allen $r$ Stellen mit $y_1,...,y_r$ enthält
    \item Falls dies der Fall ist, kam das Wort vor, gebe $1$ aus. Ansonsten 0  
  \end{enumerate}
\end{itemize}

Das obige Verfahren soll lediglich als Beispiel dienen. Es sind weitere Möglichkeiten denkbar. So können auch \textit{asymmetrische Verfahren} angewendet werden (\cite{pubenc}).

\subsection{Schutz von limitierten Geräten}

Limitierte Geräte müssen ebenfalls geschützt werden. Dabei unterscheidet man zum einen zwischen der Kommunikation zwischen den jeweiligen Gerätschaften und dem eigentlichen Geräteschutz. Im Folgenden konzentrieren wir uns auf den internen Schutz limitierter Geräte.

\subsection{Sichere Komponenten}

Bei der Entwicklung limitierter Geräte werden häufig sichere Bauteile verwendet, welche beispielsweise dafür sorgen, dass die interne Schlüsselspeicherung sicher ist und die Schlüssel auch nicht ohne weiteres gelesen werden können.
Die \autoref{fig:soc} veranschaulicht einen beispielhaften Aufbau eines limitierten Gerätes.

\begin{figure}[h]
\centering
  \includegraphics[width=0.4\textwidth]{soc}
  \caption{Diagramm eines System-on-a-Chip}
  \label{fig:soc}
\end{figure}

\subsubsection{Secure SoC} Das \textit{Secure SoC} bietet Schutz für die geräteinternen Schlüssel. Diese werden verschlüsselt im \textit{Secure Read-only-Memory} (Secure ROM) abgespeichert. Bei den Schlüsseln handelt es sich oftmals um Schlüssel, welche für digitale Signaturverfahren wie RSA, DSA oder auch ECDSA verwendet werden. Diese Schlüssel werden wie bereits gesagt verschlüsselt im ROM abgelegt. Dabei wird häufig auf Verschlüsselungsverfahren wie AES zurückgegriffen.\cite{anoopms}\\
Ein SoC bietet im besten Fall folgende Eigenschaften: 
\begin{itemize}
  \item Auf Secure ROM kann nicht physikalisch zugegriffen werden 
  \item Die Buse innerhalb des Systems können nicht abgehört werden
  \item Das Ersetzen von Komponenten soll zu einer Nichtfunktionalität führen 
\end{itemize} 

Der letzte Punkt ist dabei besonders interessant, da das Gerät dadurch mögliche Angriffe verhindern kann. Der \textit{Secure Bootloader} sorgt dabei, dass das Gerät nur mit einer korrekten Firmware hochfährt. Zusätzlich können \textit{Physical Unclonable Functions} verwendet werden, deren Funktionsweise im Folgenden erläutert wird.

\subsection{Physical Unclonable Functions}
Keine zwei Stromkreise sind identisch. Diese Idee dient als Grundlage bei der Erstellung von \textit{Physical Unclonable Functions} (PUFs). Durch Anwendung von PUFs sind Bauteile oder gesamte Geräte eindeutig identifizierbar.\\
In besten Fall sind PUFs eine Hardwareanalogon von Hashfunktionen. So gibt es im besten Fall sehr wenige Kollisionen und die Funktion selbst soll einfach durchführbar sein \cite{pufs2015}.\\
Formalisiert bedeutet das, dass es eine Challge $c$ gibt und eine Funktion $f$, die diese Challenge als Eingabe nimmt. Die Ausgabe wäre dann $r = f(c)$. 
Dabei ist zu erwähnen, dass zwei Bedingungen gelten sollen:
\begin{itemize}
  \item Für jede Challenge hat ein (integrierter) Stromkreis die gleiche Response
  \item Für jede Challenge haben unterschiedliche (integrierte) Schaltkreise unterschiedliche Responses 
\end{itemize}

Die folgende Abbildung demonstriert eine beispielhafte PUF. 

\begin{figure}[h]
\centering
  \includegraphics[width=\columnwidth]{puf}
  \caption{Multiplexer-PUF}
  \label{fig:puf}
\end{figure}

Die PUF kann wie folgt formalisiert werden: $MUX-PUF: \{0,1\}^{64} \rightarrow \{0,1\}^{64}$. \\
Sie erwartet also eine 64bit-Folge als Eingabe und produziert eine dementsprechende 64-bit-Folge als Ausgabe.\\
Diese PUF besitzt zwei verschiedene Pfade und je nachdem auf welchem Pfad das Signal schneller war, wird eine $1$ oder eine $0$ ausgegeben. Die Anzahl der Challenge-Response-Paare ist bei dem obigen beispiel $2^{64}$\cite{webpufs}.\\

Dieses Verfahren ermöglicht eine Identifikation von Komponenten (Schaltkreisen). Als Folge daraus können limitierte Geräte in einem internen Speicher die Challenge-Response-Paare für die Komponenten speichern und sich selbstständig im Betrieb überprüfen. Falls nämlich für eine Komponente kein Challenge-Response-Paar, also keine eindeutige Identifikation vorliegt, kann sich das Gerät abschalten oder gar nicht erst hochfahren (\textit{Secure Boot})\cite{suhdev}.\\
Darüber hinaus können PUFs auch zur sicheren Kommunikation zwischen Geräten verwendet werden. So können bestimmte Kommunikationspartner nur nach erfolgreicher Identifikation kommunizieren und werden andernfalls abgewiesen \cite{itseciot, eckert2015}.
% An example of a floating figure using the graphicx package.
% Note that \label must occur AFTER (or within) \caption.
% For figures, \caption should occur after the \includegraphics.
% Note that IEEEtran v1.7 and later has special internal code that
% is designed to preserve the operation of \label within \caption
% even when the captionsoff option is in effect. However, because
% of issues like this, it may be the safest practice to put all your
% \label just after \caption rather than within \caption{}.
%
% Reminder: the "draftcls" or "draftclsnofoot", not "draft", class
% option should be used if it is desired that the figures are to be
% displayed while in draft mode.
%
%\begin{figure}[!t]
%\centering
%\includegraphics[width=2.5in]{myfigure}
% where an .eps filename suffix will be assumed under latex, 
% and a .pdf suffix will be assumed for pdflatex; or what has been declared
% via \DeclareGraphicsExtensions.
%\caption{Simulation Results}
%\label{fig_sim}
%\end{figure}

% Note that IEEE typically puts floats only at the top, even when this
% results in a large percentage of a column being occupied by floats.


% An example of a double column floating figure using two subfigures.
% (The subfig.sty package must be loaded for this to work.)
% The subfigure \label commands are set within each subfloat command, the
% \label for the overall figure must come after \caption.
% \hfil must be used as a separator to get equal spacing.
% The subfigure.sty package works much the same way, except \subfigure is
% used instead of \subfloat.
%
%\begin{figure*}[!t]
%\centerline{\subfloat[Case I]\includegraphics[width=2.5in]{subfigcase1}%
%\label{fig_first_case}}
%\hfil
%\subfloat[Case II]{\includegraphics[width=2.5in]{subfigcase2}%
%\label{fig_second_case}}}
%\caption{Simulation results}
%\label{fig_sim}
%\end{figure*}
%
% Note that often IEEE papers with subfigures do not employ subfigure
% captions (using the optional argument to \subfloat), but instead will
% reference/describe all of them (a), (b), etc., within the main caption.


% An example of a floating table. Note that, for IEEE style tables, the 
% \caption command should come BEFORE the table. Table text will default to
% \footnotesize as IEEE normally uses this smaller font for tables.
% The \label must come after \caption as always.
%
%\begin{table}[!t]
%% increase table row spacing, adjust to taste
%\renewcommand{\arraystretch}{1.3}
% if using array.sty, it might be a good idea to tweak the value of
% \extrarowheight as needed to properly center the text within the cells
%\caption{An Example of a Table}
%\label{table_example}
%\centering
%% Some packages, such as MDW tools, offer better commands for making tables
%% than the plain LaTeX2e tabular which is used here.
%\begin{tabular}{|c||c|}
%\hline
%One & Two\\
%\hline
%Three & Four\\
%\hline
%\end{tabular}
%\end{table}


% Note that IEEE does not put floats in the very first column - or typically
% anywhere on the first page for that matter. Also, in-text middle ("here")
% positioning is not used. Most IEEE journals/conferences use top floats
% exclusively. Note that, LaTeX2e, unlike IEEE journals/conferences, places
% footnotes above bottom floats. This can be corrected via the \fnbelowfloat
% command of the stfloats package.



\section{Fazit}

In dieser Ausarbeitung wurden verschiedene Sicherheitsaspekte im \textit{Internet of Things} untersucht. Zu Beginn wurde auf allgemeine Sicherheitsmechanismen eingegangen, es wurden mögliche Schwachstellen und Verbesserungen genannt. Im Anschluss folgte eine Sicherheitsbetrachtung im privaten und industriellen Bereich. Es wurde deutlich, dass die Kombination zweier IKT-Bereiche neue Herausforderungen mit sich bringt, die separat bei beiden System zuvor in dieser Form nicht existierten.\\
Besonders bemerkenswert sind hierbei die unterschiedlichen Möglichkeiten, Komponenten im industriellen Bereich abzusichern. Die Möglichkeiten reichen von hardwarebasierten Lösungen wie \textit{PUFs} oder sicheren Bauteilen bis hin zu Softwarelösungen oder allgemeinen Verschlüsselungen auf der Transportebene. 



% conference papers do not normally have an appendix


% use section* for acknowledgement

\bibliographystyle{IEEEtran}
\nocite{*}
\bibliography{references}


% trigger a \newpage just before the given reference
% number - used to balance the columns on the last page
% adjust value as needed - may need to be readjusted if
% the document is modified later
%\IEEEtriggeratref{8}
% The "triggered" command can be changed if desired:
%\IEEEtriggercmd{\enlargethispage{-5in}}

% references section

% can use a bibliography generated by BibTeX as a .bbl file
% BibTeX documentation can be easily obtained at:
% http://www.ctan.org/tex-archive/biblio/bibtex/contrib/doc/
% The IEEEtran BibTeX style support page is at:
% http://www.michaelshell.org/tex/ieeetran/bibtex/
%\bibliographystyle{IEEEtran}
% argument is your BibTeX string definitions and bibliography database(s)
%\bibliography{IEEEabrv,../bib/paper}
%
% <OR> manually copy in the resultant .bbl file
% set second argument of \begin to the number of references
% (used to reserve space for the reference number labels box)




% that's all folks
\end{document}


